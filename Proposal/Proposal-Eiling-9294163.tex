% Options for packages loaded elsewhere
\PassOptionsToPackage{unicode}{hyperref}
\PassOptionsToPackage{hyphens}{url}
\PassOptionsToPackage{dvipsnames,svgnames,x11names}{xcolor}
%
\documentclass[
  12pt,
  a4paper,
]{article}

\usepackage{amsmath,amssymb}
\usepackage{setspace}
\usepackage{iftex}
\ifPDFTeX
  \usepackage[T1]{fontenc}
  \usepackage[utf8]{inputenc}
  \usepackage{textcomp} % provide euro and other symbols
\else % if luatex or xetex
  \usepackage{unicode-math}
  \defaultfontfeatures{Scale=MatchLowercase}
  \defaultfontfeatures[\rmfamily]{Ligatures=TeX,Scale=1}
\fi
\usepackage{lmodern}
\ifPDFTeX\else  
    % xetex/luatex font selection
  \setmainfont[]{Latin Modern Roman}
  \setsansfont[]{Latin Modern Roman}
\fi
% Use upquote if available, for straight quotes in verbatim environments
\IfFileExists{upquote.sty}{\usepackage{upquote}}{}
\IfFileExists{microtype.sty}{% use microtype if available
  \usepackage[]{microtype}
  \UseMicrotypeSet[protrusion]{basicmath} % disable protrusion for tt fonts
}{}
\usepackage{xcolor}
\usepackage[top=2.5cm,bottom=2.5cm,left=2.5cm,right=2.5cm]{geometry}
\setlength{\emergencystretch}{3em} % prevent overfull lines
\setcounter{secnumdepth}{-\maxdimen} % remove section numbering
% Make \paragraph and \subparagraph free-standing
\ifx\paragraph\undefined\else
  \let\oldparagraph\paragraph
  \renewcommand{\paragraph}[1]{\oldparagraph{#1}\mbox{}}
\fi
\ifx\subparagraph\undefined\else
  \let\oldsubparagraph\subparagraph
  \renewcommand{\subparagraph}[1]{\oldsubparagraph{#1}\mbox{}}
\fi


\providecommand{\tightlist}{%
  \setlength{\itemsep}{0pt}\setlength{\parskip}{0pt}}\usepackage{longtable,booktabs,array}
\usepackage{calc} % for calculating minipage widths
% Correct order of tables after \paragraph or \subparagraph
\usepackage{etoolbox}
\makeatletter
\patchcmd\longtable{\par}{\if@noskipsec\mbox{}\fi\par}{}{}
\makeatother
% Allow footnotes in longtable head/foot
\IfFileExists{footnotehyper.sty}{\usepackage{footnotehyper}}{\usepackage{footnote}}
\makesavenoteenv{longtable}
\usepackage{graphicx}
\makeatletter
\def\maxwidth{\ifdim\Gin@nat@width>\linewidth\linewidth\else\Gin@nat@width\fi}
\def\maxheight{\ifdim\Gin@nat@height>\textheight\textheight\else\Gin@nat@height\fi}
\makeatother
% Scale images if necessary, so that they will not overflow the page
% margins by default, and it is still possible to overwrite the defaults
% using explicit options in \includegraphics[width, height, ...]{}
\setkeys{Gin}{width=\maxwidth,height=\maxheight,keepaspectratio}
% Set default figure placement to htbp
\makeatletter
\def\fps@figure{htbp}
\makeatother
\newlength{\cslhangindent}
\setlength{\cslhangindent}{1.5em}
\newlength{\csllabelwidth}
\setlength{\csllabelwidth}{3em}
\newlength{\cslentryspacingunit} % times entry-spacing
\setlength{\cslentryspacingunit}{\parskip}
\newenvironment{CSLReferences}[2] % #1 hanging-ident, #2 entry spacing
 {% don't indent paragraphs
  \setlength{\parindent}{0pt}
  % turn on hanging indent if param 1 is 1
  \ifodd #1
  \let\oldpar\par
  \def\par{\hangindent=\cslhangindent\oldpar}
  \fi
  % set entry spacing
  \setlength{\parskip}{#2\cslentryspacingunit}
 }%
 {}
\usepackage{calc}
\newcommand{\CSLBlock}[1]{#1\hfill\break}
\newcommand{\CSLLeftMargin}[1]{\parbox[t]{\csllabelwidth}{#1}}
\newcommand{\CSLRightInline}[1]{\parbox[t]{\linewidth - \csllabelwidth}{#1}\break}
\newcommand{\CSLIndent}[1]{\hspace{\cslhangindent}#1}

\usepackage{fancyhdr}
\usepackage{amsmath}
\usepackage{float}
\makeatletter
\makeatother
\makeatletter
\makeatother
\makeatletter
\@ifpackageloaded{caption}{}{\usepackage{caption}}
\AtBeginDocument{%
\ifdefined\contentsname
  \renewcommand*\contentsname{Table of contents}
\else
  \newcommand\contentsname{Table of contents}
\fi
\ifdefined\listfigurename
  \renewcommand*\listfigurename{List of Figures}
\else
  \newcommand\listfigurename{List of Figures}
\fi
\ifdefined\listtablename
  \renewcommand*\listtablename{List of Tables}
\else
  \newcommand\listtablename{List of Tables}
\fi
\ifdefined\figurename
  \renewcommand*\figurename{Figure}
\else
  \newcommand\figurename{Figure}
\fi
\ifdefined\tablename
  \renewcommand*\tablename{Table}
\else
  \newcommand\tablename{Table}
\fi
}
\@ifpackageloaded{float}{}{\usepackage{float}}
\floatstyle{ruled}
\@ifundefined{c@chapter}{\newfloat{codelisting}{h}{lop}}{\newfloat{codelisting}{h}{lop}[chapter]}
\floatname{codelisting}{Listing}
\newcommand*\listoflistings{\listof{codelisting}{List of Listings}}
\makeatother
\makeatletter
\@ifpackageloaded{caption}{}{\usepackage{caption}}
\@ifpackageloaded{subcaption}{}{\usepackage{subcaption}}
\makeatother
\makeatletter
\@ifpackageloaded{tcolorbox}{}{\usepackage[skins,breakable]{tcolorbox}}
\makeatother
\makeatletter
\@ifundefined{shadecolor}{\definecolor{shadecolor}{rgb}{.97, .97, .97}}
\makeatother
\makeatletter
\makeatother
\makeatletter
\makeatother
\ifLuaTeX
  \usepackage{selnolig}  % disable illegal ligatures
\fi
\IfFileExists{bookmark.sty}{\usepackage{bookmark}}{\usepackage{hyperref}}
\IfFileExists{xurl.sty}{\usepackage{xurl}}{} % add URL line breaks if available
\urlstyle{same} % disable monospaced font for URLs
\hypersetup{
  pdftitle={Estimation of Effects of Endogenous Time-Varying Covariates: A Comparison Of Multilevel Modeling and Generalized Estimating Equations},
  pdfauthor={Ward B. Eiling (9294163)},
  colorlinks=true,
  linkcolor={blue},
  filecolor={Maroon},
  citecolor={Blue},
  urlcolor={Blue},
  pdfcreator={LaTeX via pandoc}}

\title{Estimation of Effects of Endogenous Time-Varying Covariates: A
Comparison Of Multilevel Modeling and Generalized Estimating Equations}
\usepackage{etoolbox}
\makeatletter
\providecommand{\subtitle}[1]{% add subtitle to \maketitle
  \apptocmd{\@title}{\par {\large #1 \par}}{}{}
}
\makeatother
\subtitle{PROPOSAL}
\author{Ward B. Eiling (9294163)}
\date{September 28, 2024}

\begin{document}
\cleardoublepage
\thispagestyle{empty}
{\centering
\hbox{}\vskip 0cm plus 1fill
% \vspace{25ex}
{\Large\bfseries Estimation of Effects of Endogenous Time-Varying
Covariates: A Comparison Of Multilevel Modeling and Generalized
Estimating Equations \par}
\vspace{3ex}
{\large PROPOSAL \par}
\vspace{9ex}
{\large\bfseries Ward B. Eiling (9294163) \par}
\vspace{3ex}
% {\Large ORCID: 0009-0007-8114-9497 \par}
% \vspace{3ex}
{\large Supervisors: Ellen Hamaker and Jeroen Mulder \par}
% \vskip 0cm plus 2fill
\vspace{9ex}
{\normalsize \textit{Master's degree in Methodology and Statistics for the Behavioural, \\ Biomedical and Social Sciences} \par}
\vspace{3ex}
{\normalsize \textit{Utrecht University} \par}
\vspace{9ex}
{\normalsize September 28, 2024 \par}
\vspace{3ex}
{\normalsize Word count: XXX \par}
\vspace{9ex}
{\normalsize \textit{Candidate journal: Psychological Methods} \par}
\hbox{}\vskip 0cm plus 1fill
% \vspace{12ex}
% %
% %
% {\large Utrecht University \par}
% %
% %
% {\large Methodology and Statistics \par}
% \vspace{3ex}
% %
% {\large  \par}
% %
% \vspace{12ex}
% {\small Submitted in total fulfilment of the requirements
% of the degree of Doctor of Philosophy \par}
}
\ifdefined\Shaded\renewenvironment{Shaded}{\begin{tcolorbox}[enhanced, breakable, sharp corners, frame hidden, interior hidden, borderline west={3pt}{0pt}{shadecolor}, boxrule=0pt]}{\end{tcolorbox}}\fi

\setstretch{2}
\pagestyle{fancy}
  \fancyhead{}
  \fancyhead[L]{PROPOSAL}

\floatplacement{figure}{H} % Place all figures exactly "here"

\newpage

\hypertarget{introduction}{%
\section{Introduction}\label{introduction}}

Recent trends in data-collection methods and rises in longitudinal
research have led to a proliferation of studies that employ clustered
data. To address such data, the psychological sciences most commonly
resort to multilevel linear models (MLMs), which allow us insight into
between-person heterogeneity of effects. Conversely, other fields, such
as biostatistics and econometrics often favour generalized estimating
equations (GEE, \protect\hyperlink{ref-mcneish2017}{McNeish et al.,
2017}), which does not include random effects. However, blind
application of either analysis (e.g., not for its advantages over the
other in a particular case but because of the frequency of use by fellow
researchers or by it being unknown) may cause researchers to obtain
biased estimates that do not represent the measures that they intend to
report.

Recent evidence has shed light on an issue present in both methods,
where controlling for endogenous covariates may yield biased causal
estimands (\protect\hyperlink{ref-qian2020}{Qian et al., 2020}). More
specifically, in a standard MLM with fixed covariates, coefficients may
be interpreted in the marginal (population-averaged) manner, as well as
in the conditional-on-the-random-effects manner
(\protect\hyperlink{ref-qian2020}{Qian et al., 2020, p. 3}). However,
once we include time-varying endogenous covariates, the marginal
interpretation is no longer appropriate. In a similar manner, once we
include endogenous covariates when carrying out GEE, parameter estimates
no longer follow the marginal interpretation unless the working
correlation matrix is specified as independent
(\protect\hyperlink{ref-pepe1994}{Pepe \& Anderson, 1994}). According to
Diggle (\protect\hyperlink{ref-diggle2002}{2002}), this issue not only
pertains GEE and MLM, but \emph{all} longitudinal data analysis methods.

This indicates a need to understand (1) the consequences of including
these covariates in analyses on clustered data, including GEE and
extensions of the basic MLM such as the (random-intercept) cross-lagged
panel model (CLPM) and dynamic structural equation modeling (DSEM) and
(2) how this issue may be addressed by researchers that perform
longitudinal data analysis. Accordingly, this paper explores the ways in
which the inclusion of time-varying covariates can yield faulty
inferences and intends to establish guidelines on how this may be
prevented. More specifically, the current project addresses the
following research question: \emph{to what extent does the inclusion of
endogenous variables in multilevel linear models and generalized
estimating equations result in biased estimands?} In this exploratory
study, we expect that the inclusion of endogenous time-varying
covariates elicits non-trivial bias in longitudinal data analyses that
involve a marginal interpretation.

\hypertarget{analytic-strategy}{%
\section{Analytic Strategy}\label{analytic-strategy}}

To uncover the undesirable effects of endogenous covariates and
investigate robustness against these effects, we will carry out
simulations in which data will be generated according to several
increasingly complex causal questions, which will be visually
represented using directed acyclic graphs, followed by an analysis of
these questions using the GEE, basic MLM and extensions thereof. We
consider varying number of time points and sample sizes. Statistical
analyses pertaining to the GEE and basic MLM will be performed in
\texttt{R}, version 4.2.0 (\protect\hyperlink{ref-rcoreteam2022}{R Core
Team, 2022}). To fit the GEE, the R-packages \texttt{geepack}
(\protect\hyperlink{ref-halekoh2006}{Halekoh et al., 2006}) and
\texttt{gee} (\protect\hyperlink{ref-carey2024}{Carey et al., 2024})
will evaluate several different working correlation structures,
including independent, exchangeable, AR(1) and unstructured. To fit the
basic MLM, the R-package \texttt{lme4}
(\protect\hyperlink{ref-bates2015}{Bates et al., 2015}) will be
employed, where we will both evaluate restricted maximum likelihood
estimation and ordinary maximum likelihood estimation. Extensions of the
MLM, such as the (RI-)CLPM and the bayesian DSEM will be fitted using
\texttt{Mplus}, version 8.10
(\protect\hyperlink{ref-muthuxe9n1998}{Muthén \& Muthén, 1998}).

\newpage

\hypertarget{references}{%
\section{References}\label{references}}

\hypertarget{refs}{}
\begin{CSLReferences}{1}{0}
\leavevmode\vadjust pre{\hypertarget{ref-bates2015}{}}%
Bates, D., Mächler, M., Bolker, B., \& Walker, S. (2015). Fitting linear
mixed-effects models using {lme4}. \emph{Journal of Statistical
Software}, \emph{67}(1), 148.
\url{https://doi.org/10.18637/jss.v067.i01}

\leavevmode\vadjust pre{\hypertarget{ref-carey2024}{}}%
Carey, V. J., and 4.4), T. S. L. (R. port of versions 3. 13., src/d*),
C. M. (LINPACK. routines in, \& updates), B. R. (R. port of version 4.
13. and. (2024). \emph{Gee: Generalized estimation equation solver}.
\url{https://cran.r-project.org/web/packages/gee/index.html}

\leavevmode\vadjust pre{\hypertarget{ref-Curran2007}{}}%
Curran, P. J., \& Bauer, D. J. (2007). Building path diagrams for
multilevel models. \emph{Psychological Methods}, \emph{12}(3), 283--297.
\url{https://doi.org/10.1037/1082-989X.12.3.283}

\leavevmode\vadjust pre{\hypertarget{ref-diggle2002}{}}%
Diggle, P. (2002). \emph{Analysis of Longitudinal Data}. OUP Oxford.

\leavevmode\vadjust pre{\hypertarget{ref-Drikvandi2024}{}}%
Drikvandi, R., Verbeke, G., \& Molenberghs, G. (2024). A framework for
analysing longitudinal data involving time-varying covariates. \emph{The
Annals of Applied Statistics}, \emph{18}(2), 1618--1641.
\url{https://doi.org/10.1214/23-AOAS1851}

\leavevmode\vadjust pre{\hypertarget{ref-Erler2019}{}}%
Erler, N. S., Rizopoulos, D., Jaddoe, V. W., Franco, O. H., \& Lesaffre,
E. M. (2019). Bayesian imputation of time-varying covariates in linear
mixed models. \emph{Statistical Methods in Medical Research},
\emph{28}(2), 555--568. \url{https://doi.org/10.1177/0962280217730851}

\leavevmode\vadjust pre{\hypertarget{ref-Greenland1999}{}}%
Greenland, S., Pearl, J., \& Robins, J. M. (1999). Causal diagrams for
epidemiologic research. \emph{Epidemiology}, \emph{10}(1), 37--48.
\url{https://www.jstor.org/stable/3702180}

\leavevmode\vadjust pre{\hypertarget{ref-halekoh2006}{}}%
Halekoh, U., Højsgaard, S., \& Yan, J. (2006). The r package geepack for
generalized estimating equations. \emph{Journal of Statistical
Software}, \emph{15/2}, 111.

\leavevmode\vadjust pre{\hypertarget{ref-Kim2021a}{}}%
Kim, Y., \& Steiner, P. M. (2021). Causal graphical views of fixed
effects and random effects models. \emph{British Journal of Mathematical
and Statistical Psychology}, \emph{74}(2), 165--183.
\url{https://doi.org/10.1111/bmsp.12217}

\leavevmode\vadjust pre{\hypertarget{ref-mcneish2017}{}}%
McNeish, D., Stapleton, L. M., \& Silverman, R. D. (2017). On the
unnecessary ubiquity of hierarchical linear modeling.
\emph{Psychological Methods}, \emph{22}(1), 114--140.
\url{https://doi.org/10.1037/met0000078}

\leavevmode\vadjust pre{\hypertarget{ref-Mund2021}{}}%
Mund, M., Johnson, M. D., \& Nestler, S. (2021). Changes in size and
interpretation of parameter estimates in within-person models in the
presence of time-invariant and time-varying covariates. \emph{Frontiers
in Psychology}, \emph{12}.
\url{https://doi.org/10.3389/fpsyg.2021.666928}

\leavevmode\vadjust pre{\hypertarget{ref-muthuxe9n1998}{}}%
Muthén, L. K., \& Muthén, B. O. (1998). \emph{Mplus user's guide} (Eight
Edition). Muthén \& Muthén.

\leavevmode\vadjust pre{\hypertarget{ref-pepe1994}{}}%
Pepe, M. S., \& Anderson, G. L. (1994). A cautionary note on inference
for marginal regression models with longitudinal data and general
correlated response data. \emph{Communications in Statistics -
Simulation and Computation}, \emph{23}(4), 939--951.
\url{https://doi.org/10.1080/03610919408813210}

\leavevmode\vadjust pre{\hypertarget{ref-qian2020}{}}%
Qian, T., Klasnja, P., \& Murphy, S. A. (2020). Linear mixed models with
endogenous covariates: Modeling sequential treatment effects with
application to a mobile health study. \emph{Statistical Science : A
Review Journal of the Institute of Mathematical Statistics},
\emph{35}(3), 375--390. \url{https://doi.org/10.1214/19-sts720}

\leavevmode\vadjust pre{\hypertarget{ref-rcoreteam2022}{}}%
R Core Team. (2022). \emph{R: A language and environment for statistical
computing}. \url{https://www.R-project.org/}

\leavevmode\vadjust pre{\hypertarget{ref-Robins2000}{}}%
Robins, J. M., Hernán, M. Á., \& Brumback, B. (2000). Marginal
structural models and causal inference in epidemiology.
\emph{Epidemiology}, \emph{11}(5), 550.
\url{https://journals.lww.com/epidem/fulltext/2000/09000/marginal_structural_models_and_causal_inference_in.11.aspx}

\leavevmode\vadjust pre{\hypertarget{ref-Wodtke2020}{}}%
Wodtke, G. T. (2020). Regression-based adjustment for time-varying
confounders. \emph{Sociological Methods \& Research}, \emph{49}(4),
906--946. \url{https://doi.org/10.1177/0049124118769087}

\end{CSLReferences}



\end{document}
