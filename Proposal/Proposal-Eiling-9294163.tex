% Options for packages loaded elsewhere
\PassOptionsToPackage{unicode}{hyperref}
\PassOptionsToPackage{hyphens}{url}
\PassOptionsToPackage{dvipsnames,svgnames,x11names}{xcolor}
%
\documentclass[
  12pt,
  a4paper,
]{article}

\usepackage{amsmath,amssymb}
\usepackage{setspace}
\usepackage{iftex}
\ifPDFTeX
  \usepackage[T1]{fontenc}
  \usepackage[utf8]{inputenc}
  \usepackage{textcomp} % provide euro and other symbols
\else % if luatex or xetex
  \usepackage{unicode-math}
  \defaultfontfeatures{Scale=MatchLowercase}
  \defaultfontfeatures[\rmfamily]{Ligatures=TeX,Scale=1}
\fi
\usepackage{lmodern}
\ifPDFTeX\else  
    % xetex/luatex font selection
  \setmainfont[]{Latin Modern Roman}
  \setsansfont[]{Latin Modern Roman}
\fi
% Use upquote if available, for straight quotes in verbatim environments
\IfFileExists{upquote.sty}{\usepackage{upquote}}{}
\IfFileExists{microtype.sty}{% use microtype if available
  \usepackage[]{microtype}
  \UseMicrotypeSet[protrusion]{basicmath} % disable protrusion for tt fonts
}{}
\usepackage{xcolor}
\usepackage[top=2.5cm,bottom=2.5cm,left=2.5cm,right=2.5cm]{geometry}
\setlength{\emergencystretch}{3em} % prevent overfull lines
\setcounter{secnumdepth}{5}
% Make \paragraph and \subparagraph free-standing
\ifx\paragraph\undefined\else
  \let\oldparagraph\paragraph
  \renewcommand{\paragraph}[1]{\oldparagraph{#1}\mbox{}}
\fi
\ifx\subparagraph\undefined\else
  \let\oldsubparagraph\subparagraph
  \renewcommand{\subparagraph}[1]{\oldsubparagraph{#1}\mbox{}}
\fi


\providecommand{\tightlist}{%
  \setlength{\itemsep}{0pt}\setlength{\parskip}{0pt}}\usepackage{longtable,booktabs,array}
\usepackage{calc} % for calculating minipage widths
% Correct order of tables after \paragraph or \subparagraph
\usepackage{etoolbox}
\makeatletter
\patchcmd\longtable{\par}{\if@noskipsec\mbox{}\fi\par}{}{}
\makeatother
% Allow footnotes in longtable head/foot
\IfFileExists{footnotehyper.sty}{\usepackage{footnotehyper}}{\usepackage{footnote}}
\makesavenoteenv{longtable}
\usepackage{graphicx}
\makeatletter
\def\maxwidth{\ifdim\Gin@nat@width>\linewidth\linewidth\else\Gin@nat@width\fi}
\def\maxheight{\ifdim\Gin@nat@height>\textheight\textheight\else\Gin@nat@height\fi}
\makeatother
% Scale images if necessary, so that they will not overflow the page
% margins by default, and it is still possible to overwrite the defaults
% using explicit options in \includegraphics[width, height, ...]{}
\setkeys{Gin}{width=\maxwidth,height=\maxheight,keepaspectratio}
% Set default figure placement to htbp
\makeatletter
\def\fps@figure{htbp}
\makeatother
\newlength{\cslhangindent}
\setlength{\cslhangindent}{1.5em}
\newlength{\csllabelwidth}
\setlength{\csllabelwidth}{3em}
\newlength{\cslentryspacingunit} % times entry-spacing
\setlength{\cslentryspacingunit}{\parskip}
\newenvironment{CSLReferences}[2] % #1 hanging-ident, #2 entry spacing
 {% don't indent paragraphs
  \setlength{\parindent}{0pt}
  % turn on hanging indent if param 1 is 1
  \ifodd #1
  \let\oldpar\par
  \def\par{\hangindent=\cslhangindent\oldpar}
  \fi
  % set entry spacing
  \setlength{\parskip}{#2\cslentryspacingunit}
 }%
 {}
\usepackage{calc}
\newcommand{\CSLBlock}[1]{#1\hfill\break}
\newcommand{\CSLLeftMargin}[1]{\parbox[t]{\csllabelwidth}{#1}}
\newcommand{\CSLRightInline}[1]{\parbox[t]{\linewidth - \csllabelwidth}{#1}\break}
\newcommand{\CSLIndent}[1]{\hspace{\cslhangindent}#1}

\usepackage{fancyhdr}
\usepackage{amsmath}
\usepackage{float}
\makeatletter
\makeatother
\makeatletter
\makeatother
\makeatletter
\@ifpackageloaded{caption}{}{\usepackage{caption}}
\AtBeginDocument{%
\ifdefined\contentsname
  \renewcommand*\contentsname{Table of contents}
\else
  \newcommand\contentsname{Table of contents}
\fi
\ifdefined\listfigurename
  \renewcommand*\listfigurename{List of Figures}
\else
  \newcommand\listfigurename{List of Figures}
\fi
\ifdefined\listtablename
  \renewcommand*\listtablename{List of Tables}
\else
  \newcommand\listtablename{List of Tables}
\fi
\ifdefined\figurename
  \renewcommand*\figurename{Figure}
\else
  \newcommand\figurename{Figure}
\fi
\ifdefined\tablename
  \renewcommand*\tablename{Table}
\else
  \newcommand\tablename{Table}
\fi
}
\@ifpackageloaded{float}{}{\usepackage{float}}
\floatstyle{ruled}
\@ifundefined{c@chapter}{\newfloat{codelisting}{h}{lop}}{\newfloat{codelisting}{h}{lop}[chapter]}
\floatname{codelisting}{Listing}
\newcommand*\listoflistings{\listof{codelisting}{List of Listings}}
\makeatother
\makeatletter
\@ifpackageloaded{caption}{}{\usepackage{caption}}
\@ifpackageloaded{subcaption}{}{\usepackage{subcaption}}
\makeatother
\makeatletter
\@ifpackageloaded{tcolorbox}{}{\usepackage[skins,breakable]{tcolorbox}}
\makeatother
\makeatletter
\@ifundefined{shadecolor}{\definecolor{shadecolor}{rgb}{.97, .97, .97}}
\makeatother
\makeatletter
\makeatother
\makeatletter
\makeatother
\ifLuaTeX
  \usepackage{selnolig}  % disable illegal ligatures
\fi
\IfFileExists{bookmark.sty}{\usepackage{bookmark}}{\usepackage{hyperref}}
\IfFileExists{xurl.sty}{\usepackage{xurl}}{} % add URL line breaks if available
\urlstyle{same} % disable monospaced font for URLs
\hypersetup{
  pdftitle={Estimation of Effects of Endogenous Time-Varying Covariates: A Comparison Of Multilevel Modeling and Generalized Estimating Equations},
  pdfauthor={Ward B. Eiling (9294163)},
  colorlinks=true,
  linkcolor={blue},
  filecolor={Maroon},
  citecolor={Blue},
  urlcolor={Blue},
  pdfcreator={LaTeX via pandoc}}

\title{Estimation of Effects of Endogenous Time-Varying Covariates: A
Comparison Of Multilevel Modeling and Generalized Estimating Equations}
\usepackage{etoolbox}
\makeatletter
\providecommand{\subtitle}[1]{% add subtitle to \maketitle
  \apptocmd{\@title}{\par {\large #1 \par}}{}{}
}
\makeatother
\subtitle{PROPOSAL}
\author{Ward B. Eiling (9294163)}
\date{September 28, 2024}

\begin{document}
\cleardoublepage
\thispagestyle{empty}
{\centering
\hbox{}\vskip 0cm plus 1fill
% \vspace{25ex}
{\Large\bfseries Estimation of Effects of Endogenous Time-Varying
Covariates: A Comparison Of Multilevel Modeling and Generalized
Estimating Equations \par}
\vspace{3ex}
{\large PROPOSAL \par}
\vspace{9ex}
{\large\bfseries Ward B. Eiling (9294163) \par}
\vspace{3ex}
% {\Large ORCID: 0009-0007-8114-9497 \par}
% \vspace{3ex}
{\large Supervisors: Ellen Hamaker and Jeroen Mulder \par}
% \vskip 0cm plus 2fill
\vspace{9ex}
{\normalsize \textit{Master's degree in Methodology and Statistics for the Behavioural, \\ Biomedical and Social Sciences} \par}
\vspace{3ex}
{\normalsize \textit{Utrecht University} \par}
\vspace{9ex}
{\normalsize September 28, 2024 \par}
\vspace{3ex}
{\normalsize Word count: XXX \par}
\vspace{9ex}
{\normalsize \textit{Candidate journal: Psychological Methods} \par}
\hbox{}\vskip 0cm plus 1fill
% \vspace{12ex}
% %
% %
% {\large Utrecht University \par}
% %
% %
% {\large Methodology and Statistics \par}
% \vspace{3ex}
% %
% {\large  \par}
% %
% \vspace{12ex}
% {\small Submitted in total fulfilment of the requirements
% of the degree of Doctor of Philosophy \par}
}
\ifdefined\Shaded\renewenvironment{Shaded}{\begin{tcolorbox}[frame hidden, breakable, enhanced, sharp corners, boxrule=0pt, borderline west={3pt}{0pt}{shadecolor}, interior hidden]}{\end{tcolorbox}}\fi

\setstretch{2}
\pagestyle{fancy}
  \fancyhead{}
  \fancyhead[L]{PROPOSAL}

\floatplacement{figure}{H} % Place all figures exactly "here"

\newpage

\hypertarget{introduction}{%
\section{Introduction}\label{introduction}}

\begin{itemize}
\tightlist
\item
  Start with a paragraph describing a problem in the real-life world (so
  that a BoS member not familiar with statistics understands why you are
  pursuing research in this direction);
\end{itemize}

Recent trends in data-collection methods and rises in longitudinal
research have led to a proliferation of studies that employ clustered
data. To address such data, the psychological sciences most commonly
resort to multilevel linear models (MLMs), whereas other fields, such as
biostatistics and econometrics often favour generalized estimating
equations {[}GEE; McNeish et al.
(\protect\hyperlink{ref-mcneish2017}{2017}){]}. However, blind
application of either analysis (e.g., not for its advantages over the
other in a particular case but because of the frequency of use by fellow
researchers or by it being unknown) may cause researchers to obtain
biased estimates that do not represent the measures that they intend to
report.

\begin{itemize}
\tightlist
\item
  Then, add a paragraph describing what is known in the literature;
\end{itemize}

Recent evidence has shed light on an issue present in both methods,
where controlling for covariates may yield biased causal estimands
(\protect\hyperlink{ref-qian2020}{Qian et al., 2020}). More
specificially, in a standard MLM with fixed covariates, coefficients may
be interpreted in the marginal (population-averaged) manner, as well as
in the conditional-on-the-random-effects manner
(\protect\hyperlink{ref-qian2020}{Qian et al., 2020, p. 3}). However,
once we include time-varying endogenous covariates, the marginal
interpretation is no longer appropriate. In a similar manner, once we
include endogenous covariates when carrying out GEE, parameter estimates
no longer follow the marginal interpretation unless the working
correlation matrix is specified as independent
(\protect\hyperlink{ref-pepe1994}{Pepe \& Anderson, 1994}).

\begin{itemize}
\tightlist
\item
  Also, describe what is not known and which gap you will address in
  your thesis;
\end{itemize}

\ldots{}

\begin{itemize}
\tightlist
\item
  End with a clear research question and, if applicable, your hypothesis
  (for confirmatory research questions -- it should be a testable
  hypothesis) or expectations (for exploratory research questions -- can
  be much vaguer because of the explorative nature of the question).
\end{itemize}

\ldots{}

This project aims to address and better understand the biases in the
parameter estimates of a multilevel model, when there are endogenous
time-varying covariates (e.g., Micro-randomized trials). The study will
investigate the issue introduced in Qian et al.
(\protect\hyperlink{ref-qian2020}{2020}), reframing the problem using
visualizations such as Directed Acyclic Graphs (DAGs) and path diagrams.
Additionally, it will compare the multilevel models with alternative
approaches, including Generalized Estimating Equations (GEE) and Dynamic
Structural Equation Modeling (DSEM), through simulations and empirical
analysis.

In the research report, the problem will be restated using
visualizations to enhance understanding. The biases associated with
multilevel models in the presence of endogenous time-varying confounders
will be explored, utilizing path diagrams or Directed Acyclic Graphs
(DAGs) to illustrate the problem as described by Qian et al.~(2020). To
further investigate, data will be simulated based on the descriptions
provided in the study, replicating the conditions under which
time-varying confounders and endogenous covariates affect the analysis.
The next step involves reproducing the results of multilevel model and
GEE as discussed in Qian et al.~(2020). This will include a thorough
investigation of the effects of group mean centering (GMC) and centering
within cluster (CWC) on the analysis.

In the thesis, the focus will shift to implementing solution proposed by
Qian et al.~(2020) in Dynamic Structural Equation Modeling (DSEM) using
software like Mplus. Utilizing DSEM estimation, the study will explore
(1) the potential bias in parameter estimates in multilevel models when
including endogenous covariates and (2) how interpretation of these
parameters is affected in the presence of marginal and conditional
effects. Following this, a large-scale simulation study will be
conducted to compare when GEE delivers better results than DSEM. This
study will evaluate the sensitivity of each method to model
misspecification, particularly in areas where the model is not directly
focused. We will take a multidisciplinary approach, by comparing methods
for the analysis of endogenous time-varying covariates from the fields
of biostatistics, econometrics, and social sciences. Finally, the thesis
will include a comprehensive comparative analysis of multilevel
regression, GEE, and DSEM. The advantages and disadvantages of each
method will be summarized, as well as their similarities and
differences, providing clear guidelines on their appropriate use cases.

In conclusion, the research report will provide a detailed visualization
and simulation-based exploration of the problem, while the thesis will
delve deeper into methodological comparisons and practical
implementations. Together, they will offer a comprehensive evaluation of
handling endogenous time-varying confounders, providing guidelines on
different estimation methods for researchers across different fields.

\hypertarget{analytic-strategy}{%
\section{Analytic strategy}\label{analytic-strategy}}

\begin{itemize}
\tightlist
\item
  Describe how you are planning to answer your research question and how
  to test your hypothesis or explore your question. Be as specific as
  possible and preferably use an illustration to help the reader
  understand how all your plans connect.
\item
  Also, include information on what data you are going to use and if you
  already have ethical consent. If you do already have consent, putting
  the FETC case number on your front page also suffices.
\item
  Describe what software/packages you will use.
\end{itemize}

\ldots{}

\newpage

\hypertarget{references}{%
\section{References}\label{references}}

\begin{itemize}
\tightlist
\item
  Add a reference list for the literature used in the text -- you can
  use the formatting style of the chosen journal, but this is not
  mandatory.
\item
  Add about 20 extra references that will be used for the thesis (you do
  not have to have read these already).
\end{itemize}

\hypertarget{refs}{}
\begin{CSLReferences}{1}{0}
\leavevmode\vadjust pre{\hypertarget{ref-mcneish2017}{}}%
McNeish, D., Stapleton, L. M., \& Silverman, R. D. (2017). On the
unnecessary ubiquity of hierarchical linear modeling.
\emph{Psychological Methods}, \emph{22}(1), 114--140.
\url{https://doi.org/10.1037/met0000078}

\leavevmode\vadjust pre{\hypertarget{ref-pepe1994}{}}%
Pepe, M. S., \& Anderson, G. L. (1994). A cautionary note on inference
for marginal regression models with longitudinal data and general
correlated response data. \emph{Communications in Statistics -
Simulation and Computation}, \emph{23}(4), 939--951.
\url{https://doi.org/10.1080/03610919408813210}

\leavevmode\vadjust pre{\hypertarget{ref-qian2020}{}}%
Qian, T., Klasnja, P., \& Murphy, S. A. (2020). Linear mixed models with
endogenous covariates: Modeling sequential treatment effects with
application to a mobile health study. \emph{Statistical Science : A
Review Journal of the Institute of Mathematical Statistics},
\emph{35}(3), 375--390. \url{https://doi.org/10.1214/19-sts720}

\end{CSLReferences}



\end{document}
