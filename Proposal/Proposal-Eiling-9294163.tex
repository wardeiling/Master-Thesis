% Options for packages loaded elsewhere
\PassOptionsToPackage{unicode}{hyperref}
\PassOptionsToPackage{hyphens}{url}
\PassOptionsToPackage{dvipsnames,svgnames,x11names}{xcolor}
%
\documentclass[
  12pt,
  a4paper,
]{article}

\usepackage{amsmath,amssymb}
\usepackage{setspace}
\usepackage{iftex}
\ifPDFTeX
  \usepackage[T1]{fontenc}
  \usepackage[utf8]{inputenc}
  \usepackage{textcomp} % provide euro and other symbols
\else % if luatex or xetex
  \usepackage{unicode-math}
  \defaultfontfeatures{Scale=MatchLowercase}
  \defaultfontfeatures[\rmfamily]{Ligatures=TeX,Scale=1}
\fi
\usepackage{lmodern}
\ifPDFTeX\else  
    % xetex/luatex font selection
  \setmainfont[]{Latin Modern Roman}
  \setsansfont[]{Latin Modern Roman}
\fi
% Use upquote if available, for straight quotes in verbatim environments
\IfFileExists{upquote.sty}{\usepackage{upquote}}{}
\IfFileExists{microtype.sty}{% use microtype if available
  \usepackage[]{microtype}
  \UseMicrotypeSet[protrusion]{basicmath} % disable protrusion for tt fonts
}{}
\usepackage{xcolor}
\usepackage[top=2.5cm,bottom=2.5cm,left=2.5cm,right=2.5cm]{geometry}
\setlength{\emergencystretch}{3em} % prevent overfull lines
\setcounter{secnumdepth}{-\maxdimen} % remove section numbering
% Make \paragraph and \subparagraph free-standing
\ifx\paragraph\undefined\else
  \let\oldparagraph\paragraph
  \renewcommand{\paragraph}[1]{\oldparagraph{#1}\mbox{}}
\fi
\ifx\subparagraph\undefined\else
  \let\oldsubparagraph\subparagraph
  \renewcommand{\subparagraph}[1]{\oldsubparagraph{#1}\mbox{}}
\fi


\providecommand{\tightlist}{%
  \setlength{\itemsep}{0pt}\setlength{\parskip}{0pt}}\usepackage{longtable,booktabs,array}
\usepackage{calc} % for calculating minipage widths
% Correct order of tables after \paragraph or \subparagraph
\usepackage{etoolbox}
\makeatletter
\patchcmd\longtable{\par}{\if@noskipsec\mbox{}\fi\par}{}{}
\makeatother
% Allow footnotes in longtable head/foot
\IfFileExists{footnotehyper.sty}{\usepackage{footnotehyper}}{\usepackage{footnote}}
\makesavenoteenv{longtable}
\usepackage{graphicx}
\makeatletter
\def\maxwidth{\ifdim\Gin@nat@width>\linewidth\linewidth\else\Gin@nat@width\fi}
\def\maxheight{\ifdim\Gin@nat@height>\textheight\textheight\else\Gin@nat@height\fi}
\makeatother
% Scale images if necessary, so that they will not overflow the page
% margins by default, and it is still possible to overwrite the defaults
% using explicit options in \includegraphics[width, height, ...]{}
\setkeys{Gin}{width=\maxwidth,height=\maxheight,keepaspectratio}
% Set default figure placement to htbp
\makeatletter
\def\fps@figure{htbp}
\makeatother
% definitions for citeproc citations
\NewDocumentCommand\citeproctext{}{}
\NewDocumentCommand\citeproc{mm}{%
  \begingroup\def\citeproctext{#2}\cite{#1}\endgroup}
\makeatletter
 % allow citations to break across lines
 \let\@cite@ofmt\@firstofone
 % avoid brackets around text for \cite:
 \def\@biblabel#1{}
 \def\@cite#1#2{{#1\if@tempswa , #2\fi}}
\makeatother
\newlength{\cslhangindent}
\setlength{\cslhangindent}{1.5em}
\newlength{\csllabelwidth}
\setlength{\csllabelwidth}{3em}
\newenvironment{CSLReferences}[2] % #1 hanging-indent, #2 entry-spacing
 {\begin{list}{}{%
  \setlength{\itemindent}{0pt}
  \setlength{\leftmargin}{0pt}
  \setlength{\parsep}{0pt}
  % turn on hanging indent if param 1 is 1
  \ifodd #1
   \setlength{\leftmargin}{\cslhangindent}
   \setlength{\itemindent}{-1\cslhangindent}
  \fi
  % set entry spacing
  \setlength{\itemsep}{#2\baselineskip}}}
 {\end{list}}
\usepackage{calc}
\newcommand{\CSLBlock}[1]{\hfill\break\parbox[t]{\linewidth}{\strut\ignorespaces#1\strut}}
\newcommand{\CSLLeftMargin}[1]{\parbox[t]{\csllabelwidth}{\strut#1\strut}}
\newcommand{\CSLRightInline}[1]{\parbox[t]{\linewidth - \csllabelwidth}{\strut#1\strut}}
\newcommand{\CSLIndent}[1]{\hspace{\cslhangindent}#1}

\usepackage{fancyhdr}
\usepackage{amsmath}
\usepackage{float}
\makeatletter
\@ifpackageloaded{caption}{}{\usepackage{caption}}
\AtBeginDocument{%
\ifdefined\contentsname
  \renewcommand*\contentsname{Table of contents}
\else
  \newcommand\contentsname{Table of contents}
\fi
\ifdefined\listfigurename
  \renewcommand*\listfigurename{List of Figures}
\else
  \newcommand\listfigurename{List of Figures}
\fi
\ifdefined\listtablename
  \renewcommand*\listtablename{List of Tables}
\else
  \newcommand\listtablename{List of Tables}
\fi
\ifdefined\figurename
  \renewcommand*\figurename{Figure}
\else
  \newcommand\figurename{Figure}
\fi
\ifdefined\tablename
  \renewcommand*\tablename{Table}
\else
  \newcommand\tablename{Table}
\fi
}
\@ifpackageloaded{float}{}{\usepackage{float}}
\floatstyle{ruled}
\@ifundefined{c@chapter}{\newfloat{codelisting}{h}{lop}}{\newfloat{codelisting}{h}{lop}[chapter]}
\floatname{codelisting}{Listing}
\newcommand*\listoflistings{\listof{codelisting}{List of Listings}}
\makeatother
\makeatletter
\makeatother
\makeatletter
\@ifpackageloaded{caption}{}{\usepackage{caption}}
\@ifpackageloaded{subcaption}{}{\usepackage{subcaption}}
\makeatother
\ifLuaTeX
  \usepackage{selnolig}  % disable illegal ligatures
\fi
\usepackage{bookmark}

\IfFileExists{xurl.sty}{\usepackage{xurl}}{} % add URL line breaks if available
\urlstyle{same} % disable monospaced font for URLs
\hypersetup{
  pdftitle={Estimation of Effects of Endogenous Time-Varying Covariates: A Comparison Of Multilevel Linear Modeling and Generalized Estimating Equations},
  pdfauthor={Ward B. Eiling (9294163)},
  colorlinks=true,
  linkcolor={blue},
  filecolor={Maroon},
  citecolor={Blue},
  urlcolor={Blue},
  pdfcreator={LaTeX via pandoc}}

\title{Estimation of Effects of Endogenous Time-Varying Covariates: A
Comparison Of Multilevel Linear Modeling and Generalized Estimating
Equations}
\usepackage{etoolbox}
\makeatletter
\providecommand{\subtitle}[1]{% add subtitle to \maketitle
  \apptocmd{\@title}{\par {\large #1 \par}}{}{}
}
\makeatother
\subtitle{PROPOSAL}
\author{Ward B. Eiling (9294163)}
\date{September 28, 2024}

\begin{document}
\cleardoublepage
\thispagestyle{empty}
{\centering
\hbox{}\vskip 0cm plus 1fill
% \vspace{25ex}
{\Large\bfseries Estimation of Effects of Endogenous Time-Varying
Covariates: A Comparison Of Multilevel Linear Modeling and Generalized
Estimating Equations \par}
\vspace{3ex}
{\large PROPOSAL \par}
\vspace{9ex}
{\large\bfseries Ward B. Eiling (9294163) \par}
\vspace{3ex}
% {\Large ORCID: 0009-0007-8114-9497 \par}
% \vspace{3ex}
{\large Supervisors: Ellen Hamaker and Jeroen Mulder \par}
% \vskip 0cm plus 2fill
\vspace{9ex}
{\normalsize \textit{Master's degree in Methodology and Statistics for the Behavioural, \\ Biomedical and Social Sciences} \par}
\vspace{3ex}
{\normalsize \textit{Utrecht University} \par}
\vspace{9ex}
{\normalsize September 28, 2024 \par}
\vspace{3ex}
{\normalsize Word count: 699 \par}
\vspace{9ex}
{\normalsize FETC-approved: 24-2003 \par}
\vspace{9ex}
{\normalsize \textit{Candidate journal: Psychological Methods} \par}
\hbox{}\vskip 0cm plus 1fill
% \vspace{12ex}
% %
% %
% {\large Utrecht University \par}
% %
% %
% {\large Methodology and Statistics \par}
% \vspace{3ex}
% %
% {\large  \par}
% %
% \vspace{12ex}
% {\small Submitted in total fulfilment of the requirements
% of the degree of Doctor of Philosophy \par}
}

\setstretch{2}
\pagestyle{fancy}
  \fancyhead{}
  \fancyhead[L]{PROPOSAL}

\floatplacement{figure}{H} % Place all figures exactly "here"

\newpage

\section{Introduction}\label{introduction}

Across a wide range of disciplines, researchers analyze clustered
longitudinal, observational data to investigate prospective causal
relationships between variables. When analyzing such data, the
psychological sciences most commonly resort to the multilevel linear
model (MLM, \citeproc{ref-mcneish2017}{McNeish et al., 2017}),
which---in the context of longitudinal data analysis---separates
observed variance into stable between-person differences and
within-person fluctuations (\citeproc{ref-hamaker2020}{Hamaker \&
Muthén, 2020}). Conversely, other fields, such as biostatistics and
econometrics often favour generalized estimating equations (GEE) for the
analysis of longitudinal data (\citeproc{ref-mcneish2017}{McNeish et
al., 2017}). Despite some cross-disciplinary efforts to compare these
methods (\citeproc{ref-mcneish2017}{McNeish et al., 2017};
\citeproc{ref-muth2016}{Muth et al., 2016}; \citeproc{ref-yan2013}{Yan
et al., 2013}), their scarcity may leave researchers with limited
guidance in choosing the \textbf{(1) \ldots{} most suitable approach for
their application or (2) \ldots{} right tool for the job.}

Recent evidence has highlighted an issue present in both methods, where
controlling for \emph{time-varying endogenous covariates} may lead to
biased causal estimates (\citeproc{ref-pepe1994}{Pepe \& Anderson,
1994}; \citeproc{ref-qian2020}{Qian et al., 2020}). A time-varying
covariate is \emph{endogenous} if it is directly or indirectly
influenced by prior treatment or outcome, meaning its value may be
determined by earlier stages of the process
(\citeproc{ref-qian2020}{Qian et al., 2020}). As a result of including
these covariates in the mentioned models, ordinary interpretations of
the coefficients are no longer valid (\citeproc{ref-qian2020}{Qian et
al., 2020, p. 3}). According to Diggle
(\citeproc{ref-diggle2002}{2002}), this issue not only pertains GEE and
MLM, but \emph{all} longitudinal data analysis methods.

However, due to a divide between the disciplines that employ these
methods, such critiques of the MLM appear to have failed to reach the
applied researcher in psychology. One specific reason might be that the
technical jargon in other disciplines makes it difficult for researchers
to recognize when and how these issues emerge\footnote{For instance, the
  term `endogeneity' in econometrics, while related, has a distinct
  meaning from that of an endogenous variable, which can lead to
  confusion.}. As a result, researchers may address related problems in
disconnected literatures but fail to understand each other, essentially
talking past each other (e.g., \citeproc{ref-hamaker2020}{Hamaker \&
Muthén, 2020}). For instance, while the MLM literature emphasizes on the
distinction between different centering methods and the effect of
cross-level interactions on parameter interpretations, the GEE
literature appears to focus more on the marginal and conditional
interpretations of model parameters.

Through a cross-fertilization of these debates, this project aims to (1)
explain the issue of including endogenous covariates in analyses
involving GEE, MLM and DSEM (a widely used framework in the social
sciences based on MLM) in a psychological context and (2) establish
guidelines on how researchers can prevent this issue in their
longitudinal data analysis. Accordingly, the following research question
will be addressed: \emph{to what extent does the inclusion of endogenous
variables in multilevel linear models and generalized estimating
equations result in biased estimates?} In line with the literature
(\citeproc{ref-diggle2002}{Diggle, 2002}; \citeproc{ref-pepe1994}{Pepe
\& Anderson, 1994}; \citeproc{ref-qian2020}{Qian et al., 2020}), we
expect that the inclusion of endogenous time-varying covariates in
longitudinal data analyses may result in bias that---depending on the
circumstances---can promote the potential for faulty inferences.

\newpage

\section{Analytic Strategy}\label{analytic-strategy}

To uncover the undesirable effects of endogenous covariates and
investigate robustness against these effects, we will carry out
simulations in which data will be generated according to several
increasingly complex scenarios. These scenarios will be visually
represented using directed acyclic graphs and analyzed using GEE, MLM
and DSEM. We will start out with a scenario of the basic MLM---where a
time-varying outcome \(Y\) is regressed on a single time-varying
predictor \(X\) and in the presence of stable between person differences
in the intercept---and increase the complexity until we reach the
scenario that includes a time-varying endogenous covariate. The primary
interest of this simulation study is the comparative performance of
different specifications of the MLM, GEE and DSEM in terms of bias in
the estimation of the effect of \(X\) to \(Y\). The secondary interest
is the efficiency in mean squared error (MSE). We consider varying
number of time points and sample sizes.

Statistical analyses pertaining to the GEE and basic MLM will be
performed in \texttt{R}, version 4.2.0 (\citeproc{ref-rcoreteam2022}{R
Core Team, 2022}). To fit the GEE, the R-packages \texttt{geepack}
(\citeproc{ref-halekoh2006}{Halekoh et al., 2006}) and \texttt{gee}
(\citeproc{ref-carey2024}{Carey et al., 2024}) will evaluate several
different working correlation structures, including independent,
exchangeable, AR(1) and unstructured. To fit the basic MLM, the
R-package \texttt{lme4} (\citeproc{ref-bates2015}{Bates et al., 2015})
will be employed, where we will both evaluate restricted maximum
likelihood estimation and ordinary maximum likelihood estimation.
Extensions of the MLM from the DSEM framework will be fitted using
\texttt{Mplus}, version 8.10 (\citeproc{ref-muthuxe9n1998}{Muthén \&
Muthén, 1998}).

\newpage

\section{References}\label{references}

\phantomsection\label{refs}
\begin{CSLReferences}{1}{0}
\bibitem[\citeproctext]{ref-bates2015}
Bates, D., Mächler, M., Bolker, B., \& Walker, S. (2015). Fitting linear
mixed-effects models using {lme4}. \emph{Journal of Statistical
Software}, \emph{67}(1), 148.
\url{https://doi.org/10.18637/jss.v067.i01}

\bibitem[\citeproctext]{ref-carey2024}
Carey, V. J., and 4.4), T. S. L. (R. port of versions 3. 13., src/d*),
C. M. (LINPACK. routines in, \& updates), B. R. (R. port of version 4.
13. and. (2024). \emph{Gee: Generalized estimation equation solver}.
\url{https://cran.r-project.org/web/packages/gee/index.html}

\bibitem[\citeproctext]{ref-Curran2007}
Curran, P. J., \& Bauer, D. J. (2007). Building path diagrams for
multilevel models. \emph{Psychological Methods}, \emph{12}(3), 283--297.
\url{https://doi.org/10.1037/1082-989X.12.3.283}

\bibitem[\citeproctext]{ref-diggle2002}
Diggle, P. (2002). \emph{Analysis of Longitudinal Data}. OUP Oxford.

\bibitem[\citeproctext]{ref-Drikvandi2024}
Drikvandi, R., Verbeke, G., \& Molenberghs, G. (2024). A framework for
analysing longitudinal data involving time-varying covariates. \emph{The
Annals of Applied Statistics}, \emph{18}(2), 1618--1641.
\url{https://doi.org/10.1214/23-AOAS1851}

\bibitem[\citeproctext]{ref-Erler2019}
Erler, N. S., Rizopoulos, D., Jaddoe, V. W., Franco, O. H., \& Lesaffre,
E. M. (2019). Bayesian imputation of time-varying covariates in linear
mixed models. \emph{Statistical Methods in Medical Research},
\emph{28}(2), 555--568. \url{https://doi.org/10.1177/0962280217730851}

\bibitem[\citeproctext]{ref-Greenland1999}
Greenland, S., Pearl, J., \& Robins, J. M. (1999). Causal diagrams for
epidemiologic research. \emph{Epidemiology}, \emph{10}(1), 37--48.
\url{https://www.jstor.org/stable/3702180}

\bibitem[\citeproctext]{ref-halekoh2006}
Halekoh, U., Højsgaard, S., \& Yan, J. (2006). The r package geepack for
generalized estimating equations. \emph{Journal of Statistical
Software}, \emph{15/2}, 111.

\bibitem[\citeproctext]{ref-hamaker2020}
Hamaker, E. L., \& Muthén, B. (2020). The fixed versus random effects
debate and how it relates to centering in multilevel modeling.
\emph{Psychological Methods}, \emph{25}(3), 365--379.
\url{https://doi.org/10.1037/met0000239}

\bibitem[\citeproctext]{ref-Kim2021a}
Kim, Y., \& Steiner, P. M. (2021). Causal graphical views of fixed
effects and random effects models. \emph{British Journal of Mathematical
and Statistical Psychology}, \emph{74}(2), 165--183.
\url{https://doi.org/10.1111/bmsp.12217}

\bibitem[\citeproctext]{ref-mcneish2017}
McNeish, D., Stapleton, L. M., \& Silverman, R. D. (2017). On the
unnecessary ubiquity of hierarchical linear modeling.
\emph{Psychological Methods}, \emph{22}(1), 114--140.
\url{https://doi.org/10.1037/met0000078}

\bibitem[\citeproctext]{ref-Mund2021}
Mund, M., Johnson, M. D., \& Nestler, S. (2021). Changes in size and
interpretation of parameter estimates in within-person models in the
presence of time-invariant and time-varying covariates. \emph{Frontiers
in Psychology}, \emph{12}.
\url{https://doi.org/10.3389/fpsyg.2021.666928}

\bibitem[\citeproctext]{ref-muth2016}
Muth, C., Bales, K. L., Hinde, K., Maninger, N., Mendoza, S. P., \&
Ferrer, E. (2016). Alternative Models for Small Samples in Psychological
Research: Applying Linear Mixed Effects Models and Generalized
Estimating Equations to Repeated Measures Data. \emph{Educational and
Psychological Measurement}, \emph{76}(1), 64--87.
\url{https://doi.org/10.1177/0013164415580432}

\bibitem[\citeproctext]{ref-muthuxe9n1998}
Muthén, L. K., \& Muthén, B. O. (1998). \emph{Mplus user's guide} (Eight
Edition). Muthén \& Muthén.

\bibitem[\citeproctext]{ref-pepe1994}
Pepe, M. S., \& Anderson, G. L. (1994). A cautionary note on inference
for marginal regression models with longitudinal data and general
correlated response data. \emph{Communications in Statistics -
Simulation and Computation}, \emph{23}(4), 939--951.
\url{https://doi.org/10.1080/03610919408813210}

\bibitem[\citeproctext]{ref-qian2020}
Qian, T., Klasnja, P., \& Murphy, S. A. (2020). Linear mixed models with
endogenous covariates: Modeling sequential treatment effects with
application to a mobile health study. \emph{Statistical Science : A
Review Journal of the Institute of Mathematical Statistics},
\emph{35}(3), 375--390. \url{https://doi.org/10.1214/19-sts720}

\bibitem[\citeproctext]{ref-rcoreteam2022}
R Core Team. (2022). \emph{R: A language and environment for statistical
computing}. \url{https://www.R-project.org/}

\bibitem[\citeproctext]{ref-raudenbush2002}
Raudenbush, S. W., \& Bryk, A. S. (2002). \emph{Hierarchical Linear
Models: Applications and Data Analysis Methods} (2nd ed.). SAGE.

\bibitem[\citeproctext]{ref-Robins2000}
Robins, J. M., Hernán, M. Á., \& Brumback, B. (2000). Marginal
structural models and causal inference in epidemiology.
\emph{Epidemiology}, \emph{11}(5), 550.
\url{https://journals.lww.com/epidem/fulltext/2000/09000/marginal_structural_models_and_causal_inference_in.11.aspx}

\bibitem[\citeproctext]{ref-Wodtke2020}
Wodtke, G. T. (2020). Regression-based adjustment for time-varying
confounders. \emph{Sociological Methods \& Research}, \emph{49}(4),
906--946. \url{https://doi.org/10.1177/0049124118769087}

\bibitem[\citeproctext]{ref-yan2013}
Yan, J., Aseltine, R. H., \& Harel, O. (2013). Comparing Regression
Coefficients Between Nested Linear Models for Clustered Data With
Generalized Estimating Equations. \emph{Journal of Educational and
Behavioral Statistics}, \emph{38}(2), 172--189.
\url{https://doi.org/10.3102/1076998611432175}

\end{CSLReferences}



\end{document}
